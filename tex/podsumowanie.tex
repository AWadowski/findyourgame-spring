\chapter*{Podsumowanie}
\addcontentsline{toc}{chapter}{Podsumowanie}

W trakcie realizacji tej pracy dokonano szczegółowego porównania dwóch różnych baz danych: relacyjnej bazy Oracle oraz nierelacyjnej bazy Firebase. Analizy te miały miejsce w kontekście ich integracji z aplikacją webową napisaną w technologii Spring Boot oraz frontendem napisanym przy użyciu Angulara, służącą do wyszukiwania gier według zadanych preferencji użytkownika.

Jednym z najbardziej znaczących odkryć jest wyraźna przewaga Firebase pod względem wydajności i szybkości w stosunku do Oracle. W przypadku różnorodnych zapytań i transakcji, Firebase znacząco przyspieszył czas odpowiedzi, co jest kluczowe dla zapewnienia dobrej jakości użytkowania w aplikacjach o dużym natężeniu ruchu i dynamicznie zmieniających się danych.

Firebase wykazał się również dużą elastycznością w obszarze skalowania, zarówno wertykalnego jak i horyzontalnego. To czyni go idealnym wyborem dla start-upów oraz dla aplikacji, które doświadczają dynamicznego wzrostu liczby użytkowników czy też zmiennego wolumenu danych. Jego struktura oparta na JSON-ie ułatwia również pracę z różnymi formami nierelacyjnych danych, co może być atutem w wielu nowoczesnych aplikacjach webowych i mobilnych.

Nie można jednak pominąć, że Oracle z jego rozbudowanym zestawem funkcji i mechanizmów zapewniających konsystencję i trwałość danych, pozostaje solidnym wyborem dla zastosowań korporacyjnych, gdzie te cechy są niezbędne. Wszakże, w kontekście badanej aplikacji, te atuty nie były wystarczająco przekonujące, aby zrównoważyć różnice w wydajności.

Co więcej, warto zwrócić uwagę na różnice w kosztach i złożoności administracyjnej. Firebase, będąc rozwiązaniem chmurowym, oferuje znaczne ułatwienia w zarządzaniu i utrzymaniu bazy, co również może wpłynąć na końcowy wybór, szczególnie dla mniejszych zespołów deweloperskich lub indywidualnych programistów.

W kontekście tych odkryć, praca ta dostarcza wartościowych wskazówek dla praktyków i teoretyków w dziedzinie baz danych. Ostateczny wybór bazy danych powinien być dokonany po uwzględnieniu szeregu kryteriów, takich jak wydajność, skalowalność, koszt, oraz specyficzne potrzeby projektu. Jednak, w kontekście badanego przypadku i kryteriów wydajności, Firebase prezentuje się jako zdecydowanie lepsza opcja.