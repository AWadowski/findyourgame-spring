\chapter*{Wstęp}
\addcontentsline{toc}{chapter}{Wstęp}


Wielu programistów na początku budowy projektu zderza się z problemem wyboru odpowiedniej bazy danych. Programiści oczekują często jak najmniejszego czasu oczekiwania na wołane przez nich zapytania, łatwej obsługi oraz dobrej integracji z ich aplikacjami, które tworzą do użytku masowego przez wielu użytkowników. Bazy danych zwiększają swoją objętość, a  ich przeszukiwanie staje się coraz dłuższe. Osobiście borykałem się z tym wyborem już wiele razy, podczas budowy aplikacji w trakcie studiów, życia prywatnego, a także zawodowego. Najczęstszym problemem pojawiającym się na samym początku jest rozpoznanie, która baza danych będzie najbardziej efektywna dla naszej aplikacji, czy będzie to baza relacyjna, czy może nierelacyjna. Jednakże na początku drogi programisty nikt nie jest w stanie dokonać najlepszego wyboru bez odpowiedniego rozeznania.

Zdecydowanie odpowiedniej bazy danych może mieć znaczący wpływ na ogólną wydajność, skalowalność i elastyczność aplikacji. Wybór jest szczególnie trudny, gdy stajemy przed decyzją pomiędzy bazą danych relacyjną a nierelacyjną. Obydwa rodzaje baz danych mają swoje zalety i~wady, które mogą być kluczowe w kontekście różnych zastosowań i wymagań.

W tej pracy porównam dwie różne bazy danych: relacyjną bazę danych Oracle i nierelacyjną bazę danych Firebase. Oceniane będą one w kontekście ich integracji z aplikacją webową, stworzoną przy użyciu technologii Spring Boot, służącą do wyszukiwania gier wideo na podstawie zadanych preferencji użytkownika. Zostaną poddane analizie różne aspekty, takie jak czas odpowiedzi na zapytania, łatwość implementacji, koszty utrzymania oraz skalowalność.

Zarówno Oracle, jako przykład bazy relacyjnej, jak i Firebase, będąc reprezentantem baz nierelacyjnych są powszechnie używane w różnorodnych aplikacjach. Poprzez porównanie ich wydajności i funkcjonalności w realnym scenariuszu użycia, ta praca ma na celu dostarczenie praktycznych wskazówek, które pomogą programistom w podjęciu świadomej decyzji odnośnie wyboru bazy danych dla ich aplikacji.

W pierwszym rozdziale zostanie przedstawiony proces tworzenia aplikacji webowej. Na podstawie tego procesu zostaną omówione podstawowe błędy, problemy oraz ich rozwiązania w trakcie tworzenia aplikacji. Poza tworzeniem aplikacji zostanie poruszony również temat porównania bazy relacyjnej i nierelacyjnej pod kątem odpowiedniego wykorzystania.

W kolejnym rozdziale zostaną przedstawione technologie użyte do utworzenia aplikacji, która będzie służyła do mierzenia wydajności baz danych poprzez porównywanie czasu wykonywania zapytań. Omówione zostaną technologie użyte przy tworzeniu \textit{backendu} jak i \textit{frontendu}. Technologie zostaną przeanalizowane pod kątem wad i zalet oraz przypadków użycia.

W trzecim rozdziale opisane zostanie zastosowanie technologii podanych w poprzednim rozdziale. Przedstawiony będzie kod aplikacji oraz omówienie poszczególnych funkcjonalności, które prowadzą do zbierania danych badawczych.

Ostatni rozdział zostanie poświęcony omówieniu wyników badań. Zostaną  wyciągnięte wnioski oraz sformułowane podsumowanie odpowiadające na róznicę w wydajności baz relacyjnych i~nierelacyjnych. Mam nadzieję, że ta praca będzie użytecznym
źródłem informacji dla programistów na różnych etapach kariery, zarówno dla tych, którzy dopiero
zaczynają swoją przygodę z~bazami danych, jak i dla doświadczonych deweloperów, poszukujących
optymalizacji ich obecnych rozwiązań.
