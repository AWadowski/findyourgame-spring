\chapter*{Wstęp}
\addcontentsline{toc}{chapter}{Wstęp}


DO WYCIĘCIA/PRZEROBIENIA

Wielu programistów na początku budowy projektu, zderza się z problemem wyboru najbardziej wydajnej bazy danych. Programiści oczekują często jak najmniejszego czasu oczekiwania na wołane przez nich zapytania, łatwej obsługi oraz dobrej integracji z ich aplikacjami, które tworzą do użytku masowego przez wielu użytkowników, przez co bazy danych zwiększają swoją objętość, a  ich przeszukiwanie staje się coraz dłuższe. Osobiście borykałem się z tym wyborem już wiele razy, podczas budowy aplikacji w trakcie studiów, życia prywatnego, a także zawodowego. Najczęstszym problemem pojawiącym się na samym początku, jest rozpoznanie, która baza danych będzie najbardziej optymalna dla naszej aplikacji, czy będzie to baza relacyjna, czy może nierelacyjna, jednakże na początku drogi programisty nikt nie jest w stanie dokonać najlepszego wyboru bez odpowiedniego rozeznania i pomocy od osób trzecich.

Celem przedstawionej pracy, jest pokazanie różnic w tworzeniu, implementacji oraz wydajności bazy relacyjnej i nierelacyjnej, na podstawie aplikacji webowej pisanej z użyciem techonologii Spring Boot. 
