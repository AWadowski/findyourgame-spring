\chapter*{Wstęp}
\addcontentsline{toc}{chapter}{Wstęp}


Wielu programistów na początku budowy projektu zderza się z problemem wyboru najbardziej wydajnej bazy danych. Programiści oczekują często jak najmniejszego czasu oczekiwania na wołane przez nich zapytania, łatwej obsługi oraz dobrej integracji z ich aplikacjami, które tworzą do użytku masowego przez wielu użytkowników. Bazy danych zwiększają swoją objętość, a  ich przeszukiwanie staje się coraz dłuższe. Osobiście borykałem się z tym wyborem już wiele razy, podczas budowy aplikacji w trakcie studiów, życia prywatnego, a także zawodowego. Najczęstszym problemem pojawiącym się na samym początku jest rozpoznanie, która baza danych będzie najbardziej efektywna dla naszej aplikacji, czy będzie to baza relacyjna, czy może nierelacyjna. Jednakże na początku drogi programisty nikt nie jest w stanie dokonać najlepszego wyboru bez odpowiedniego rozeznania i pomocy od osób trzecich.

Zdecydowanie odpowiedniej bazy danych może mieć znaczący wpływ na ogólną wydajność, skalowalność i elastyczność aplikacji. Wybór jest szczególnie trudny, gdy stajemy przed decyzją pomiędzy bazą danych relacyjną a nierelacyjną. Obydwa rodzaje baz danych mają swoje zalety i wady, które mogą być kluczowe w kontekście różnych zastosowań i wymagań.

W tej pracy porównam dwie różne bazy danych: relacyjną bazę danych Oracle i nierelacyjną bazę danych Firebase. Oceniane będą one w kontekście ich integracji z aplikacją webową, stworzoną przy użyciu technologii Spring Boot, służącą do wyszukiwania gier wideo na podstawie zadanych preferencji użytkownika. Zostaną poddane analizie różne aspekty, takie jak czas odpowiedzi na zapytania, łatwość implementacji, koszty utrzymania oraz skalowalność.

Zarówno Oracle, jako przykład bazy relacyjnej, jak i Firebase, będąc reprezentantem baz nierelacyjnych, są powszechnie używane w różnorodnych aplikacjach. Poprzez bezpośrednie porównanie ich wydajności i funkcjonalności w realnym scenariuszu użycia, ta praca ma na celu dostarczenie praktycznych wskazówek, które pomogą programistom w podjęciu świadomej decyzji odnośnie wyboru bazy danych dla ich aplikacji.

Dokładne analizy i testy zostaną przedstawione w kolejnych rozdziałach, a ich wyniki posłużą do sformułowania końcowych wniosków i rekomendacji. Mam nadzieję, że ta praca będzie użytecznym źródłem informacji dla programistów na różnych etapach kariery, zarówno dla tych, którzy dopiero zaczynają swoją przygodę z bazami danych, jak i dla doświadczonych deweloperów, poszukujących optymalizacji ich obecnych rozwiązań.
