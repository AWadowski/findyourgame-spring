\chapter{Technologie i języki programowania wykorzystane do stworzenia aplikacji webowej}

\section{Java}
\begin{figure}[h]
    \centering
    \includegraphics[width=0.7\linewidth]{./img/javalogo.jpg}
    \caption{Logo języka Java \cite{java}}
    \label{fig:Java}
\end{figure}
Java to język programowania wysokiego poziomu oraz platforma obliczeniowa, która została zaprojektowana w połowie lat 90. przez firmę Sun Microsystems. Od tego czasu Java stała się jednym z najbardziej popularnych języków programowania na świecie. Java została stworzona przez grupę inżynierów pod kierownictwem Jamesa Goslinga w Sun Microsystems w 1991 roku jako część projektu Green, który miał na celu rozwój inteligentnych urządzeń do domu. Jednakże, zamiast tego, Java znalazła swoje miejsce w świecie internetu. Pierwsza oficjalna wersja (Java 1.0) została wydana w 1996 roku.

Charakterystyczne cechy:
\begin{itemize}
\item \textbf{Przenośność:} Dzięki maszynie wirtualnej Java (JVM), kod napisany w Javie jest przenośny i może być uruchamiany na różnych platformach bez konieczności modyfikacji.
\item \textbf{Bezpieczeństwo:} Java oferuje różne mechanizmy bezpieczeństwa, takie jak zarządzanie pamięcią, które chronią przed wieloma popularnymi błędami programistycznymi.
\item \textbf{Wielowątkowość:} Java obsługuje programowanie wielowątkowe, co pozwala na równoczesne wykonywanie wielu zadań.
\item \textbf{Obiektowość:} Java jest językiem w pełni obiektowym, co ułatwia organizację i strukturyzację kodu.
\end{itemize}

Język ten ma bardzo bogate zastosowanie:
\begin{itemize}
\item \textbf{Aplikacje webowe:} Java jest często używana do tworzenia serwerów aplikacji, serwisów RESTful czy aplikacji opartych o mikroserwisy.
\item \textbf{Aplikacje mobilne:} Java jest głównym językiem programowania dla systemu Android.
\item \textbf{Aplikacje desktopowe:} Za pomocą JavaFX czy Swing można tworzyć bogate interfejsy użytkownika.
\item \textbf{Systemy wbudowane i IoT:} Java jest używana w urządzeniach wbudowanych, takich jak telewizory, samochody czy różnego rodzaju sensory.
\item \textbf{Aplikacje korporacyjne:} Java jest często wybierana do tworzenia dużych, rozbudowanych systemów korporacyjnych ze względu na jej wydajność i skalowalność.
\end{itemize}

Aby dobrze wykorzystać potencjał tego języka należy zwrócić uwagę na następujące zalecenia:
\begin{itemize}
\item \textbf{Ucz się bibliotek:} Java ma ogromną bibliotekę standardową oraz wiele zewnętrznych bibliotek. Znajomość tych narzędzi może znacząco przyspieszyć proces tworzenia aplikacji.
\item \textbf{Dbaj o jakość kodu:} Java jest językiem, który łatwo się komplikuje. Regularne przeglądy kodu, testy jednostkowe i stosowanie wzorców projektowych mogą pomóc w utrzymaniu kodu w dobrej kondycji.
\item \textbf{Bądź na bieżąco:} Java jest ciągle rozwijana. Nowe wersje przynoszą nowe funkcje, które mogą ułatwić i przyspieszyć pracę.
\end{itemize}

\section{Spring Boot}
\begin{figure}[h]
    \centering
    \includegraphics[width=0.6\linewidth]{./img/springboot.png}
    \caption{Logo frameworku Spring Boot \cite{springboot}}
    \label{fig:Springboot}
\end{figure}
Spring Boot, będący kluczowym elementem ekosystemu Spring, to nowoczesny framework zaprojektowany z myślą o uproszczeniu procesu tworzenia aplikacji opartych na Springu. Jego głównym celem jest eliminacja konieczności ręcznej konfiguracji, co pozwala programistom skupić się na tworzeniu funkcjonalności aplikacji i szybkim wdrażaniu jej.

\begin{itemize}
\item \textbf{Automatyzacja konfiguracji:} Spring Boot automatycznie konfiguruje aplikację na podstawie dostępnych w projekcie bibliotek. Dzięki temu programiści mogą skupić się na kodzie, nie martwiąc się o skomplikowane ustawienia.
\item \textbf{Starters:} Są to zestawy zależności, które upraszczają dodawanie funkcjonalności do aplikacji. Na przykład, chcąc dodać wsparcie dla bazy danych MongoDB, wystarczy dodać odpowiedni "starter", a Spring Boot zajmie się resztą.

\item \textbf{Narzędzia produkcyjne:} Spring Boot zawiera narzędzia przeznaczone do monitorowania i zarządzania aplikacją w środowisku produkcyjnym, takie jak monitorowanie metryk, przeglądanie logów czy analiza stanu aplikacji w czasie rzeczywistym.

\item \textbf{Zalety Spring Boot:}
\begin{itemize}
    \item \textbf{Szybkość tworzenia:} Automatyzacja konfiguracji i dostępność "starters" przyspieszają proces tworzenia aplikacji.
    \item \textbf{Mikroserwisy:} Doskonała współpraca z Spring Cloud ułatwia tworzenie architektury opartej na mikroserwisach.
    \item \textbf{Integracja z bazami danych:} Wsparcie dla wielu popularnych baz danych, zarówno relacyjnych, jak i nierelacyjnych.
    \item \textbf{Bezpieczeństwo:} Łatwa integracja z Spring Security umożliwia dodawanie funkcji bezpieczeństwa do aplikacji.
\end{itemize}

\item \textbf{Zastosowania:} Spring Boot jest niezwykle wszechstronny. Wykorzystywany jest do tworzenia aplikacji biznesowych, systemów e-commerce, aplikacji korporacyjnych oraz w rozwiązaniach opartych o mikroserwisy. Jego elastyczność i łatwość użycia sprawiają, że jest jednym z najpopularniejszych frameworków w świecie Javy.
\end{itemize}

\section{Lombok}
\begin{figure}[h]
    \centering
    \includegraphics[width=0.4\linewidth]{./img/lombok.png}
    \caption{Logo biblioteki Lombok \cite{lombok}}
    \label{fig:Lombok}
\end{figure}
Biblioteka Lombok to narzędzie dla języka Java, które znacząco upraszcza proces tworzenia kodu poprzez eliminację powtarzalnych fragmentów, takich jak gettery, settery, konstruktory czy metody hashCode() i equals(). Te często powtarzające się fragmenty są nieodłącznym elementem tradycyjnych aplikacji Java, ale dzięki Lombokowi można je zastąpić kilkoma adnotacjami, co sprawia, że kod staje się bardziej zwięzły i czytelny.

Główne cechy i zalety Lomboka:

\begin{itemize}
\item \textbf{Automatyczna generacja kodu:} Lombok automatycznie generuje kod dla wielu powszechnie używanych funkcji, takich jak gettery, settery, konstruktory czy metody toString(). Wystarczy dodać odpowiednią adnotację do klasy lub pola, a Lombok zajmie się resztą.

\item \textbf{Zwięzłość:} Dzięki Lombokowi, klasy stają się znacznie krótsze i bardziej zrozumiałe. Na przykład, zamiast ręcznie pisać cały kod dla gettera i settera, można po prostu użyć adnotacji @Getter i @Setter.

\item \textbf{Redukcja błędów}: Ręczne pisanie powtarzalnego kodu jest podatne na błędy. Lombok redukuje ryzyko wprowadzenia błędów poprzez automatyzację tego procesu.

\item \textbf{Integracja z IDE:} Popularne środowiska programistyczne, takie jak IntelliJ IDEA czy Eclipse, oferują wsparcie dla Lomboka, co ułatwia pracę z tą biblioteką.

\item \textbf{Elastyczność:} Lombok oferuje wiele adnotacji, które można dostosować do indywidualnych potrzeb. Na przykład, można kontrolować, które pola są uwzględniane w generowanych metodach czy jakie modyfikatory dostępu mają generowane metody.

\item \textbf{Wsparcie dla wzorców projektowych:} Lombok ułatwia implementację niektórych wzorców projektowych, takich jak wzorzec Singleton, poprzez dostarczenie dedykowanych adnotacji, takich jak @Singleton.
\end{itemize}

Przykłady użycia Lomboka:

\begin{itemize}
\item \textbf{@Data:} Jest to jedna z najbardziej wszechstronnych adnotacji w Lomboku. Generuje gettery, settery, hashCode(), equals() oraz toString() dla całej klasy.
\item \textbf{@Slf4j:} Dodaje loggera do klasy, co jest
\end{itemize} przydatne w aplikacjach korzystających z logowania.
Mimo wielu zalet, warto również pamiętać o pewnych ograniczeniach i potencjalnych problemach związanych z używaniem Lomboka, takich jak kompatybilność z niektórymi narzędziami czy trudności w debugowaniu automatycznie wygenerowanego kodu. Niemniej jednak, dla wielu programistów korzyści płynące z użycia Lomboka przeważają nad jego wadami.

\section{Liquibase}
\begin{figure}[h]
    \centering
    \includegraphics[width=0.4\linewidth]{./img/liquibase.png}
    \caption{Logo narzędzia Liquibase \cite{liquibase}}
    \label{fig:Liquibase}
\end{figure}
Liquibase to otwarte narzędzie do zarządzania i śledzenia zmian w bazie danych. Umożliwia programistom kontrolę wersji schematu bazy danych, co jest niezbędne w dynamicznie rozwijających się aplikacjach, gdzie struktura bazy danych może ulegać częstym modyfikacjom.

Główne cechy i zalety Liquibase:

\begin{itemize}
\item \textbf{Kontrola wersji schematu bazy danych:} Podobnie jak systemy kontroli wersji dla kodu źródłowego, Liquibase pozwala śledzić, dokumentować i cofać zmiany w bazie danych.

\item \textbf{Niezmienność:} Każda zmiana jest traktowana jako nieodwracalna. Oznacza to, że raz zastosowana migracja nie jest modyfikowana, co zapewnia spójność i powtarzalność procesu aktualizacji.

\item \textbf{Formaty opisu zmian:} Zmiany w bazie danych można opisywać w różnych formatach, takich jak XML, YAML, JSON czy SQL.

\item \textbf{Nie zależy od bazy danych:} Liquibase został zaprojektowany tak, aby działać z wieloma systemami baz danych. Dzięki temu można używać tego samego narzędzia niezależnie od wybranej technologii bazy danych.

\item \textbf{Integracja z narzędziami budowy:} Liquibase łatwo integruje się z popularnymi narzędziami do budowy i wdrażania aplikacji, takimi jak Maven, Gradle czy Jenkins.

\item \textbf{Wsparcie dla środowisk wielu deweloperów:} Wspiera scenariusze, w których wielu deweloperów pracuje nad jednym projektem, pomagając rozwiązywać konflikty i zapewniając spójność schematu bazy danych.
\end{itemize}
Przykłady użycia Liquibase:

\begin{itemize}
\item \textbf{Aktualizacja schematu:} Jeśli deweloper wprowadza zmiany w schemacie bazy danych, może je opisać w pliku zmian Liquibase, a następnie zastosować te zmiany w bazie danych za pomocą narzędzia.

\item \textbf{Rollback:} Jeśli wprowadzona zmiana powoduje problemy, Liquibase umożliwia łatwe cofnięcie tej zmiany.
\end{itemize}
Liquibase, podobnie jak każde narzędzie, ma pewne ograniczenia. Może wymagać pewnego nakładu pracy przy konfiguracji, a także pewnej krzywej uczenia się, zwłaszcza dla tych, którzy nie są zaznajomieni z kontrolą wersji dla baz danych. Jedną z wielu wad przy budowie tej aplikacji było brak integracji z Firebase. Niemniej jednak, dla wielu zespołów deweloperskich korzyści płynące z użycia Liquibase, takie jak spójność, powtarzalność i kontrola nad zmianami w bazie danych, przeważają nad potencjalnymi wadami.

\section{TypeScript}
\begin{figure}[h]
    \centering
    \includegraphics[width=0.5\linewidth]{./img/typescript.png}
    \caption{Logo języka TypeScript \cite{typescript}}
    \label{fig:TypeScript}
\end{figure}
TypeScript to język programowania opracowany przez Microsoft, który rozszerza JavaScript o opcjonalne typowanie statyczne i inne funkcje. Jego głównym celem jest ułatwienie tworzenia dużych i złożonych aplikacji w JavaScript, zapewniając narzędzia i funkcje, które pomagają w identyfikacji błędów na wczesnym etapie procesu deweloperskiego.

Główne cechy i zalety TypeScript:
\begin{itemize}
\item \textbf{Opcjonalne typowanie statyczne:} Jedną z głównych cech TypeScript jest możliwość dodawania opcjonalnych typów do zmiennych, argumentów funkcji i wartości zwracanych. Pomaga to w wykrywaniu błędów typów na etapie kompilacji.

\item \textbf{Wsparcie dla najnowszych funkcji ECMAScript:} TypeScript obsługuje najnowsze funkcje i składnię ECMAScript, a także umożliwia kompilację kodu do starszych wersji JavaScript, co jest przydatne dla zachowania kompatybilności z różnymi środowiskami.

\item \textbf{Interfejsy i klasy:} TypeScript wprowadza koncepcję interfejsów i klas, które pomagają w organizacji kodu i tworzeniu bardziej modularnych i skalowalnych aplikacji.

\item \textbf{Dekoratory i metadane:} TypeScript oferuje dekoratory, które pozwalają na dodawanie metadanych do klas, metod i właściwości, co jest przydatne w wielu scenariuszach, takich jak programowanie sterowane aspektami.

\item \textbf{Narzędzia deweloperskie:} Dzięki integracji z popularnymi środowiskami IDE, takimi jak Visual Studio Code, TypeScript oferuje zaawansowane funkcje, takie jak podświetlanie składni, autouzupełnianie kodu i nawigacja po kodzie źródłowym.

\item \textbf{Wsparcie dla typów zewnętrznych:} Społeczność TypeScript dostarcza definicje typów dla wielu popularnych bibliotek JavaScript, co ułatwia ich używanie w projektach TypeScript.
\end{itemize}

Przykłady użycia TypeScript:

\begin{itemize}
\item \textbf{Aplikacje jednostronicowe (SPA):} TypeScript jest często używany do tworzenia zaawansowanych aplikacji internetowych, takich jak Angular, React czy Vue.

\item \textbf{Aplikacje serwerowe:} Dzięki integracji z Node.js, TypeScript jest również używany do tworzenia aplikacji serwerowych.
\end{itemize}

Wprowadzenie TypeScript do istniejącego projektu JavaScript może wymagać pewnych modyfikacji w kodzie. Ponadto, choć opcjonalne typowanie jest jednym z głównych atutów TypeScript, może również wprowadzić pewną złożoność, zwłaszcza dla tych, którzy są nowi w świecie silnie typowanych języków. Niemniej jednak, dla wielu deweloperów korzyści płynące z użycia TypeScript, takie jak lepsza organizacja kodu, łatwiejsze debugowanie i większa produktywność, przeważają nad potencjalnymi wadami.

\section{Angular}
\begin{figure}[h]
    \centering
    \includegraphics[width=0.6\linewidth]{./img/angular.png}
    \caption{Logo frameworku Angular \cite{angular}}
    \label{fig:Angular}
\end{figure}
Angular to popularny framework do tworzenia aplikacji internetowych opracowany i utrzymywany przez Google. Został zaprojektowany z myślą o tworzeniu dynamicznych, jednostronicowych aplikacji internetowych (SPA) i oferuje zestaw narzędzi do efektywnego zarządzania danymi, logiką biznesową i interfejsem użytkownika.

Główne cechy i zalety Angulara:
\begin{itemize}
\item \textbf{Komponentowy system architektury:} Angular opiera się na komponentach, które są niezależnymi jednostkami logiki i interfejsu użytkownika. Umożliwia to modularność, łatwe testowanie i ponowne użycie kodu.

\item \textbf{Dwukierunkowe wiązanie danych (Two-way data binding):} Umożliwia automatyczną synchronizację pomiędzy modelem a widokiem, co sprawia, że aktualizacje w jednym miejscu są natychmiast odzwierciedlane w drugim.

\item \textbf{Wsparcie dla SPA:} Angular został zaprojektowany z myślą o tworzeniu jednostronicowych aplikacji internetowych, które oferują płynne przejścia między widokami bez konieczności przeładowywania całej strony.

\item \textbf{Zaawansowany routing:} Angular oferuje potężny system routingu, który pozwala na ładowanie komponentów w oparciu o stan URL, leniwe ładowanie modułów i zagnieżdżone widoki.

\item \textbf{Wsparcie dla formularzy:} Angular dostarcza narzędzia do tworzenia reaktywnych i szablonowych formularzy, które ułatwiają walidację i obsługę danych wejściowych.

\item \textbf{Wbudowane narzędzia do testowania:} Angular zawiera narzędzia do jednostkowego i integracyjnego testowania aplikacji, co ułatwia zapewnienie jakości kodu.

\item \textbf{Wsparcie dla programowania reaktywnego:} Dzięki integracji z biblioteką RxJS, Angular umożliwia tworzenie reaktywnych aplikacji opartych na strumieniach danych.
\end{itemize}

Przykłady użycia Angulara:
\begin{itemize}
\item \textbf{Aplikacje korporacyjne:} Dzięki swojej skalowalności i wsparciu dla modułowości, Angular jest często wybierany do tworzenia dużych aplikacji korporacyjnych.

\item \textbf{Platformy e-commerce:} Angular oferuje narzędzia do tworzenia dynamicznych sklepów internetowych z zaawansowanymi funkcjami.

\item \textbf{Aplikacje mobilne:} Za pomocą narzędzi takich jak Ionic, Angular może być również używany do tworzenia aplikacji mobilnych.
\end{itemize}

Krzywa uczenia się Angulara jest stosunkowo stroma, zwłaszcza dla tych, którzy są nowi w świecie frameworków front-endowych. Ponadto, choć Angular jest bardzo wszechstronny, może być "zbyt ciężki" dla prostych projektów, gdzie lżejsze rozwiązania, takie jak React czy Vue, mogą być bardziej odpowiednie. Niemniej jednak, dla wielu deweloperów korzyści płynące z użycia Angulara, takie jak jego wszechstronność, wsparcie społeczności i ciągłe aktualizacje, przeważają nad potencjalnymi wadami.

\section{PrimeNG}
\begin{figure}[h]
    \centering
    \includegraphics[width=0.5\linewidth]{./img/primeng.png}
    \caption{Logo narzędzia PrimeNG \cite{primeng}}
    \label{fig:PrimeNG}
\end{figure}

PrimeNG to zestaw komponentów interfejsu użytkownika dla Angulara. Opracowany przez PrimeTek, PrimeNG dostarcza bogatą kolekcję gotowych do użycia komponentów, które ułatwiają szybkie tworzenie zaawansowanych aplikacji internetowych z wykorzystaniem Angulara.

Główne cechy i zalety PrimeNG:

\begin{itemize}
\item \textbf{Bogata kolekcja komponentów:} PrimeNG oferuje szeroką gamę komponentów, od podstawowych elementów, takich jak przyciski czy listy, po zaawansowane komponenty, takie jak wykresy, drzewa czy tabelki z funkcją paginacji.


\item \textbf{Tematyzacja:} PrimeNG dostarcza zestaw gotowych motywów, które pozwalają na szybkie dostosowanie wyglądu aplikacji do indywidualnych potrzeb. Dodatkowo, dzięki wsparciu dla narzędzia theming API, użytkownicy mogą tworzyć własne motywy.

\item \textbf{Wsparcie dla mobilności:} Komponenty PrimeNG są responsywne i dostosowują się do różnych rozmiarów ekranów, co czyni je odpowiednimi zarówno dla aplikacji desktopowych, jak i mobilnych.

\item \textbf{Integracja z Angulara:} Jako że PrimeNG został zaprojektowany specjalnie dla Angulara, jego komponenty doskonale integrują się z tym frameworkiem, oferując spójne API i wydajność.

\item \textbf{Wysoka wydajność:} Komponenty PrimeNG są zoptymalizowane pod względem wydajności, co zapewnia płynne działanie nawet w dużych i złożonych aplikacjach.

\item \textbf{Wsparcie społeczności i dokumentacja:} PrimeNG posiada aktywną społeczność, która regularnie dzieli się wskazówkami i rozwiązaniami problemów. Ponadto, biblioteka ta jest dobrze udokumentowana, co ułatwia jej wdrożenie i użycie.
\end{itemize}

Przykłady użycia PrimeNG:

\begin{itemize}
\item \textbf{Zaawansowane panele administracyjne:} Dzięki szerokiej gamie komponentów, takich jak tabele, wykresy czy formularze, PrimeNG jest idealnym wyborem do tworzenia zaawansowanych paneli administracyjnych.

\item \textbf{Aplikacje biznesowe:} PrimeNG jest często wykorzystywany w aplikacjach korporacyjnych, gdzie potrzebne są zaawansowane komponenty interfejsu użytkownika.

\item \textbf{Platformy e-commerce:} Komponenty takie jak karuzele produktów, listy rozwijane czy kalendarze czynią PrimeNG atrakcyjnym wyborem dla platform e-commerce.
\end{itemize}

PrimeNG może wymagać pewnego nakładu pracy przy konfiguracji, a niektóre komponenty mogą nie być dostosowane do bardzo specyficznych potrzeb. Niemniej jednak, dla wielu deweloperów korzyści płynące z użycia PrimeNG, takie jak bogata kolekcja komponentów, wsparcie społeczności i ciągłe aktualizacje, przeważają nad potencjalnymi wadami.

\section{Podsumowanie}

Wybór odpowiednich technologii i narzędzi jest jednym z najważniejszych etapów w procesie tworzenia aplikacji. Decyzje te mają bezpośredni wpływ na wydajność, skalowalność i utrzymanie projektu w przyszłości. Omówione w tym rozdziale technologie, takie jak Java, Spring Boot, Lombok, Liquibase, TypeScript, Angular i PrimeNG, stanowią przykład zaawansowanych i sprawdzonych rozwiązań, które są szeroko stosowane w branży IT. Każde z nich przynosi unikalne korzyści, które mogą znacząco przyczynić się do sukcesu projektu.

Jednakże, niezależnie od zalet poszczególnych technologii, kluczem jest ich prawidłowe zastosowanie i integracja. Ważne jest, aby zrozumieć specyfikę projektu, jego wymagania oraz oczekiwania użytkowników. Tylko wtedy można dokonać świadomego wyboru, który przyniesie oczekiwane korzyści.

Współczesny świat technologii oferuje nieskończone możliwości. Dlatego też, nieustanne kształcenie się, eksplorowanie nowych narzędzi i adaptacja do zmieniającego się środowiska są kluczem do tworzenia innowacyjnych i skutecznych rozwiązań. W końcu to nie tylko technologia, ale przede wszystkim ludzie i ich wizja, determinują sukces każdego projektu.