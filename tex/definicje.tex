\documentclass[10pt, a4paper]{book}
\pagestyle{plain}
\usepackage[a4paper,left=3cm,right=3cm,top=3cm,bottom=3cm]{geometry}
\geometry{a4paper}

\usepackage{listings}

\lstset{
    % language=Pascal,
    basicstyle=\small,
    showstringspaces=false,
    numbers=left,
    numberstyle=\footnotesize,
    stepnumber=1,
    numbersep=5pt,
    showspaces=false,
    showtabs=false,
    frame=single,
    tabsize=2,
    captionpos=b,
    breaklines=true,
    breakatwhitespace=false,
    %keywordstyle=\ttfamily\color{green}
    %identifierstyle=\ttfamily\color{yellow}\bfseries,
    %commentstyle=\color{brown},
    stringstyle=\ttfamily,
    escapeinside={\%*}{*)},
    inputencoding=utf8,
    literate={ą}{{\k{a}}}1 {Ą}{{\k{A}}}1 {ę}{{\k{e}}}1 {Ę}{{\k{E}}}1 {ó}{{\'o}}1 {Ó}{{\'O}}1 {ś}{{\'s}}1 {Ś}{{\'S}}1 {ł}{{\l{}}}1 {Ł}{{\L{}}}1 {ż}{{\.z}}1 {Ż}{{\.Z}}1 {ź}{{\'z}}1 {Ź}{{\'Z}}1 {ć}{{\'c}}1 {Ć}{{\'C}}1 {ń}{{\'n}}1 {Ń}{{\'N}}1
}
\usepackage[utf8]{inputenc}
\usepackage{amsmath}
\usepackage{amssymb}
\usepackage{amsthm}
\usepackage{amsbsy}
\usepackage{latexsym}
\usepackage{indentfirst}
\usepackage[onehalfspacing]{setspace}
\usepackage{marvosym}
\usepackage{listings}
\usepackage{fancyvrb}
%\usepackage{clrscode}
\usepackage{amsmath, amsthm, amssymb}
\usepackage{amsfonts}
\usepackage[polish]{babel}
\usepackage[OT4]{fontenc}
\usepackage{graphicx}
\usepackage{epstopdf}
\usepackage{subfig}
\usepackage{wrapfig}
\usepackage{longtable}
\usepackage{verbatim}
\usepackage{float}
\usepackage{psfrag}
\usepackage{makecell}
\usepackage{wrapfig}
\usepackage{array}
\usepackage{url}
\linespread{1}

\DeclareGraphicsRule{.tif}{png}{.png}{`convert #1 `dirname #1`/`basename #1 .tif`.png}

\clubpenalty=3000
\widowpenalty=600
\hyphenpenalty=8000
\sloppy

\newcommand{\FullFileListing}[1]{\VerbatimInput[frame=lines,label={[Plik #1]Koniec pliku #1},framesep=1em,numbers=left,numbersep=2pt,fontsize=\footnotesize]{../application/#1}}
\DefineVerbatimEnvironment{pyline}{Verbatim}{frame=lines,fontsize=\footnotesize}
\DefineVerbatimEnvironment{pylinex}{Verbatim}{frame=leftline,fontsize=\footnotesize}
\DefineVerbatimEnvironment{pylinen}{Verbatim}{frame=lines,fontsize=\footnotesize,numbers=left}
\newcommand{\zlamLinie}{\Rewind{}}

\newcounter{defscounter}
\setcounter{tocdepth}{1}

\theoremstyle{definition}
\newtheorem{odesolution}[defscounter]{Definicja}
\newtheorem{odeexample}{Przykład}

\newtheorem{hyperbolicPDE}[defscounter]{Definicja}

\renewcommand{\lstlistlistingname}{Spis listingów}
\renewcommand{\lstlistingname}{Listing}

\newenvironment{changemargin}[1]{%
\begin{list}{}{%
\setlength{\topsep}{0pt}%
\setlength{\leftmargin}{#1}%
\setlength{\rightmargin}{\parindent}%
\setlength{\listparindent}{\parindent}%
\setlength{\itemindent}{\parindent}%
\setlength{\parsep}{\parskip}%
}%
\item[]}{\end{list}}

\makeatletter
 \renewcommand\@seccntformat[1]{\csname the#1\endcsname. }
 \renewcommand\numberline[1]{#1.\hskip0.7em}
\makeatother

